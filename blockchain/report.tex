% !Mode\dots ``TeX:UTF-8''
\documentclass[serif]{beamer}
%\usetheme{Warsaw}
\usepackage{hyperref}
\usepackage{subcaption}
\usepackage{euscript}
%\usepackage{natbib}
\DeclareGraphicsExtensions{.png,.jpg}
\usepackage{default}


%自定义的宏
\include{def}
% 图片路径
\graphicspath{{img/}}
\title{BlockChain}
\subtitle{network flows problems}
\author{ Liyun Dai}

\institute{RISE, Southwest University, Chongqing, China}
\date{\today}

\begin{document}
\maketitle
\begin{frame}
  \frametitle{Table of Contents}
  \tableofcontents[currentsection]
\end{frame}

\AtBeginSection[]
{\begin{frame}
	\frametitle{Table of Contents}
		\tableofcontents[currentsection]
\end{frame}}
\section{INTRODUCTION}

\begin{frame}{Original problem}
	\begin{problem}
		The original problem comes from the idea to have some
		thing represented as {\color{red}{digital entity}} that can be passed {\color{red}{securely}}
		from one party to another.
	\end{problem}
\end{frame}

\begin{frame}{Example}
	\begin{example}
		Imagine you have some money in
		your bank account, and you would like to securely transfer
		some of it to another person
	\end{example}
	
	{\Huge Blank is a trusting center.}
\end{frame}

\begin{frame}{Example 	without bank }

	\begin{problem}
		Yes it is possible. However, regardless of representation any digital sequence can be copied.
		This creates the well-known problem of {\color{red}{“double-spending”}}:
		one person makes an electronic transaction more than once
		using the same money.
	\end{problem}
\end{frame}
\section{HASH FUNCTIONS}
\begin{frame}{Hash function}
	\begin{definition}
		Hash function is a method to convert data of arbitrary size
		into a digital string of predefined {\color{red}{fixed length}}, called hash.
	\end{definition}
\end{frame}
\begin{frame}{Hash function}
		\begin{definition}[cryptographic hash functions]
			It should be easy to compute the hash, but finding any input to the hash
			function {\color{red}{must be difficult}}, or better virtually impossible. Hash functions with such a property are sometimes called
			cryptographic hash functions.
		\end{definition}
\end{frame}
\begin{frame}
	The best known cryptographic hash functions are
	{\tt MD5}, {\tt SHA1}, {\tt SHA2}. 
	\begin{example}
		An example of {\tt MD5} is this
		{\tt MD5}(``abc") = 900150983cd24fb0d6963f7d28e17f72
	\end{example}
\end{frame}

\end{document}